\section{Cardinalities}

It is common to place conditions on the number of nodes in a set of values.
Let $x \in TERM$ be a node, let $q \in Path$ be a path, and let $n \in \nat$ be a natural number.
Let $minCount(x,q,n)$ be the set of graphs $g$ in which $values(x,q,g)$ has at least $n$ nodes.
\begin{axdef}
minCount: ParameterizedConstraint[TERM \cross PathEx \cross \nat]
\where
minCount = (\lambda x: TERM; q: PathEx; n: \nat @ \\
\t1	\{~ g: Graph | \# (values(g,x,q)) \geq n ~\})
\end{axdef}

Similarly, $maxCount(x,q,n)$ is the set of graphs $g$ in which $values(x,q,g)$ has at most $n$ nodes.
\begin{axdef}
maxCount: ParameterizedConstraint[TERM \cross PathEx \cross \nat]
\where
maxCount = (\lambda x: TERM; q: PathEx; n: \nat @ \\
\t1	\{~ g: Graph | \# (values(g,x,q)) \leq n ~\})
\end{axdef}

\section{The OSLC Resource Shape Language}

The OSLC Resource Shape specification defines an RDF-based shape language and the interpretation of this language as graph constraints. This document provides a formal specification of the interpretation of the OSLC shape language.

The shape language expresses several kinds of constraint.
Each kind of constraint may be optionally used in any given shape instance.
Optional values are modelled as sets that contain either zero or one member.
\begin{zed}
OPTIONAL[X] == \{~ x: \finset X | \# x \leq 1 ~\}
\end{zed}

The shape language is capable of expressing constraints on the
set of values associated with a path from a given node.
A path shape consists of optional min and max value occurrence constraints, and an optional reference to a resource shape that constrains the values.
\begin{schema}{PathShape}
minOccurs: OPTIONAL[\nat] \\
maxOccurs: OPTIONAL[\nat] \\
shape: OPTIONAL[IRI]
\end{schema}

The shape language is capable of expressing constraints associated with a given resource.
A resource shape consists of a finite set of paths and for each path, a path shape.
A resource shape refers to a finite, possibly empty, set of resource shapes.
\begin{schema}{ResourceShape}
lookup: PathEx \pfun PathShape
\also
paths: \finset PathEx \\
shapes: \finset IRI
\where
paths = \dom lookup
\also
shapes = \bigcup \{~ q: paths @ (lookup~q).shape ~\}
\end{schema}
\begin{itemize}
\item Each path has an associated path shape.
\item The set of resource shapes referred to by this resource shape is the union of the resource shapes referred to
by its path shapes.
\end{itemize}

Let $GraphShape$ denote the OSLC shape language. 
A sentence in the language consists of a starting node $node$, the name of the
starting resource shape $shape$, and a finite lookup table $lookup$ that maps names to resource shapes.
\begin{schema}{GraphShape}
node: TERM \\
shape: IRI \\
lookup: IRI \pfun ResourceShape
\also
shapes: \finset IRI
\where
node \notin Literal
\also
shapes = \dom lookup
\also
shape \in shapes
\also
\forall r: shapes @ (lookup~r).shapes \subseteq shapes
\end{schema}
\begin{itemize}
\item The starting node is not a literal.
\item The set of all resource shape names in the lookup table is $shapes$.
\item The starting resource shape is in the lookup table.
\item The lookup table contains all referenced resource shapes.
\end{itemize}

\section{Interpretation of the OSLC Resource Shape Language as Constraints}

Given an instance $gs \in GraphShape$ of the OSLC resource shape language, its interpretation as a graph
constraint is given by $matches(gs)$.
\begin{axdef}
matches: ParameterizedConstraint[GraphShape]
\end{axdef}

The definition of $matches(gs)$ is given in terms of the interpretations of the resource and path shapes contained in $gs$.
This definition is necessarily recursive because a path shape may refer to a resource shape.
Therefore, $matches$ will be defined axiomatically and we will show that the definition defines $matches$ uniquely.

\begin{axdef}
matchesMinOccurs: ParameterizedConstraint[TERM \cross PathEx \cross PathShape]
\where
\forall x: TERM; q: PathEx; ps: PathShape | \\
\t1	ps.minOccurs = \emptyset @ \\
\t2		matchesMinOccurs(x,q,ps) = Graph
\also
\forall x: TERM; q: PathEx; ps: PathShape | \\
\t1	ps.minOccurs = \emptyset @ \\
\t2		matchesMinOccurs(x,q,ps) = Graph
\end{axdef}

